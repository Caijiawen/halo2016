\chapter{绪论}
\section{意义和背景}
1.regularized logistic + feature engineer
2.twitter api + outlier detection + fraud detection
\section{文献综述}



\subsection{数据科学在保险业中的应用}
李娜娜(2013),介绍了数据挖掘基本理论,对医疗保险进行了需求分析,介绍了数据仓库的设计以及数据的结构及存储,分别通过聚类进行区别定价,决策树进行客户风险控制,神经网络进行欺诈案件识别。
Varun Chandola(2008)描述了健康保险理赔数据的类型和特征,指出在健康保险中的三种问题:欺诈,浪费,滥用,使用文本挖掘(LDA主题模型识别欺诈模式),社交网络分析,序列分析识别欺诈并提高健康保险运作效率。

\subsection{医疗风险评级}
Moturu(2009)\cite{Moturu2009}采用非随机抽样平衡敏感性(Sensitivity)与特异性(Specificity),并在Adaboost,SVM等算法中实现以牺牲少量特异性的情况下大幅提高敏感性。




\cite{Moturu2009}采用非随机抽样平衡敏感性(Sensitivity)与特异性(Specificity)